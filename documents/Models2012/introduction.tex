\section{Introduction}

Modeling frameworks (e.g. the Eclipse Modeling Framework (EMF)~\cite{emf2009} or Kermeta~\cite{kermeta}) can only work with a model when it is fully loaded into a computer's main memory (RAM), even though not all model objects are used at the same time.
This limits the possible size of a model. 
Modeling frameworks themselves provide only limited capabilities to deal with large models (i.e. resources and resource lazy loading in EMF~\cite{emf2009}). 
Model persistence frameworks (e.g. Connected Data Objects (CDO)~\cite{cdo}), on the other hand, store models in databases and load and unload single model objects on demand. 
Only those objects that are used at the same time need to be maintained in main memory at the same time. This allows to work with models larger than the main memory would otherwise allow. 

We claim that existing model persistence solutions may provide a main memory efficient solution to the model size issue, but not a time efficient one. In this paper, time efficiency always relates the time it takes to execute of one of four abstract modeling tasks. These tasks are (i) creating/modify models, (ii) traverse models (e.g. as necessary during model transformation), (iii) query models, and (iv) partially loading models (i.e. loading a diagram into an editor).

An obvious observation is that some of these modeling tasks (especially traversing models and loading parts of models) require to load large numbers of model objects eventually. Existing persistence frameworks, store and access model objects individually. If a tasks requires to load a larger part of the model, all its objects are still accessed individually from the underlying database. This is time consuming.

Our hypothesis is that modeling tasks can be executed faster, if models are mapped to larger aggregates within an underlying database. Storing models as aggregates of objects and not as single objects reduces the number of required database accesses, or as Martin Fowler puts it on his blog: \emph{"Storing aggregates as fundamental units makes a lot of sense} [...]\emph{, since you have a large clump of data that you expect to be accessed together"},~\cite{martinFowler}. This hypothesis raises three major questions: Do models contain aggregates that are often \emph{accessed together}? How can we determine aggregates automatically and transparently? What actual influence on the performance has the choice of concrete aggregates?

To answer these question, we will proceed as follows: First (section~\ref{sec:applications}), we look at three typical modeling applications: which model sizes they work with and what concrete modeling tasks they perform predominantly. This will give us an idea of what aggregates could be and how often objects can be expected to be actually accessed as aggregates. 
Secondly (section~\ref{sec:fragmentation}), we will present our approach to finding aggregates within models. This approach is based on fragmenting models along their containment hierarchy. We will reason, that most modeling tasks need to access sub-trees of the containment hierarchy (fragments). 
In the related work section~\ref{sec:related_work}, we present existing model persistence frameworks and interpret their strategies with respect to the idea of fragmentation. Furthermore, we discuss key-value stores as a basis for persisting fragmented models.
The following section provides a theoretical analysis and upper bound estimation for possible performance gains with optimal fragmentation.
In section~\ref{sec:implemention}, we finally present a framework that implements our fragmentation concept.
The next section is the evaluation section: we compare our framework to existing persistence frameworks with respect to time and memory efficient execution of the four mentioned abstract modeling tasks. Furthermore, we use our framework to measure the influence of fragmentation on performance to verify the analytic considerations from section~\ref{sec:fragmentation}.
We close the paper with further work and conclusions. 

