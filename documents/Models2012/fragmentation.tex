\section{Model Fragmentation}

\subsection{Fragmention in General}
All models considered in this paper have an inherent containment hierarchy. This means the models are graphs, but within these graphs there is a single distinct subset of edges that spans a tree. In EMF ecore based models, these trees consist of containment references. Hence, we also call these spanning trees containment trees.

In the large section, we saw that typical model applications mostly use aggregates of model objects, more precisely sub-trees of the model's containment trees. The model fragmentation approach, divides (i.e. \emph{fragments}) a model along its containment tree. All \emph{fragments} are disjoint; no object is part of two fragments. Fragmentation is also always complete, i.e. each object is part of one fragment. The set of fragments of a model is called \emph{fragmentation}. Based on these characteristics, fragments can be compared to EMF's resources (especially with containment proxies); refer to section~\ref{sec:implementation}, where we use resources to realize fragments.

\begin{figure}[ht]
\begin{minipage}[b]{0.5\linewidth}
\centering
\includegraphics[width=\linewidth]{figures/metamodel_fragmentation_patterns_tree}
\caption{An example pattern in a model and a fragmentation. The red links are inter fragment links.}
\label{fig:metamodel_fragmentation_pattern_tree}
\end{minipage}
\hspace{0.5cm}
\begin{minipage}[b]{0.5\linewidth}
\centering
\includegraphics[width=\linewidth]{figures/metamodel_fragmentation_patterns_graph}
\caption{Another example pattern. For a single feature there are two types of instances: inter- and intra-fragment links.}
\label{fig:metamodel_fragmentation_pattern_graph}
\end{minipage}
\end{figure}  

Fig.~\ref{fig:metamodel_fragmentation_pattern_tree} and Fig.~\ref{fig:metamodel_fragmentation_pattern_graph} present two examples.

\subsection{Fragmentation Strategies}

Of course a model has no fragmentation originally, further do we need to maintain a fragmentation when the model is manipulated, and we need to assume that fragmentation has an influence on performance. We denote algorithms necessary to create and maintain a fragmentation as \emph{fragmentation strategies}.

There are two trivial strategies: \emph{no fragmentation} and \emph{total fragmentation}. No fragmentation means the whole model constitutes the one and only fragment, such as in regular EMF (without resources). Total fragmentation means each object constitutes its own fragment. There are as many fragments as objects in the model.

There are other strategies imaginable. We will analyse one in more detail. 

\subsection{Meta-Model based Fragmentation}

A meta-model defines possible models (graph of objects and links between objects) by means of classes and associations (i.e. structural features in EMF ecore). A good meta-model only allows models that are suitable for the use-cases of the envision model application. Containment references (indirectly the containment hierarchy) are already used by the meta-modeller to aggregate (e.g. in UML composition is a special case of aggregation) closely related objects. For the \emph{meta-model based fragmentation} fragmentation strategy, we ask the meta-modeller to determine inter-fragment reference features as specific containment references. This way, the meta-model determines where the spanning-tree is broken into fragments. 

Only containment reference features can be designated as inter-fragment reference features and all instances of such a reference feature will be inter-fragment references (like a-references in Fig.~\ref{fig:metamodel_fragmentation_pattern_tree}). Other references (i.e. cross refernces) can become inter-fragment references \emph{by accident} (like c-references in Fig.~\ref{fig:metamodel_fragmentation_pattern_graph}).

One the inter-fragment reference features are designated within the meta-model, it is easy to create and maintain fragmentations automatically and transparently; ref. to section~\ref{sec:implementation}.

%\subsubsection{Theory}
%Access patterns for a model are strongly influenced by its metamodel.
%Metamodels are tiny in comparison to their large instances. 
%If you imagine looking from above onto a large model, the metamodel types of its objects form patterns. How we access a model is also influenced by its metamodel, since all algorithms doing something with a model are programmed against its metamodel.
%Hence, optimal fragmentation goes along this patterns.
%Most fragments will have the same structure, and fragments are connected through structural features of only few different types.
%One way to define fragmentation is to mark these fragment crossing structural features. 
%
%Fig.~\ref{fig:metamodel_fragmentation_pattern_tree} shows a simple example metamodel type pattern. The instances (links) of feature \emph{a} cross fragment borders (inter fragment links). All other links are intra-fragment links. The situation is a  little more complicated in Fig.~\ref{fig:metamodel_fragmentation_pattern_graph}. Here the links of feature \emph{c} have both inter- and intra-fragment instances. 
%
%If we want to describe fragmentation by marking features as inter- or intra-fragment features, it would work for the example in Fig.~\ref{fig:metamodel_fragmentation_pattern_tree}, but not for the example in Fig.~\ref{fig:metamodel_fragmentation_pattern_graph}. Obviously, we need further restrictions.
%
%Models (as used in this paper) always have a inherent spanning tree. The spanning tree is formed from links that are instances from containment features. All instances of containment features are part of the spanning tree. If we only allow containment features to be inter-fragment features then the instances of inter-fragment feature will always define a unique fragmentation. \markus{Proof?}
%
%\subsubsection{Implementation}
%This describes an implementation based on EMF. 
%
%\subsection{Automated Fragmentation based on Expected Range Queries}
%\subsection{Automated Fragmentation based on Access Patterns}
%\subsection{Fragmentation of Even Models}
%
%\subsubsection{Analysis}
%
%At the beginning, we will look at \emph{even} models. A model is even, if its inner structure suggest fragmentation into equal pieces. For example, an intuitive way to fragment a OO software model is to put each package into one entry. This is an uneven model, since packages have different sizes. Another example is sensor data, sensor data produced at each point in time or on each node has the same size. If one puts each sensor reading or each node into one entry, the entries will have similar size.
%
%Previously, we were looking the gains achieved with optimal fragmentation. While optimal fragmentation is plausible in manually fragmented models for a single specific loaded model (e.g. accessing single sensor readings in ClickWatch). Optimal fragmentation is unlikely for different loaded models (even impossible for models of different size). 
%
%In general, we can assume that the smaller $ope$ is compared to $load$, the more likely it is that much of each entry is part of the loaded model. In other words, the smaller my entries are, the more likely it is that much or all if a single entry is part of the loaded model. We will model $part$  accordingly.