\section{Conclusions}\label{sec:conclusions}

\begin{itemize}

\item We proposed model fragmentation for model data base persistence with better performance characteristics that existing solutions (e.g. CDO, Morsa).

\item For most modelling tasks (create, traverse, query, partial load) fragmented models perform better or similar to existing solutions. For create, traverse, and partial load performance enhancements are of magnitudes.

\item Fragmentation can be achieved with meta-model based fragmentations, which already provides close to optimal performance for many applications and tasks (e.g. editing and compiling for software code models). Other strategies are imaginable and future work.

\item The downside is that the underlying key-value stores (generally) do not provide a transaction mechanism. Future work: implement transaction within EMFFrag or use key-value stores with transactions (e.g. Scalaris).

\item On the up-side, good scalarbility for both model size and demand for parallel access: e.g. in geo-spatial models. Good integration into parallel computing, i.e. with map-reduce frameworks.

\end{itemize}